%%%%%
%%
%% Notebooks live in this directory.  There are two kinds: in-game
%% notebooks (like classic research notebooks) and out-of-game
%% notebooks (also known as green notebooks).  Both kinds are defined
%% in Lists/notebook-LIST.tex.  In-game notebooks, defined as Notebook
%% macros, get assigned in \MYitems.  Out-of-game notebooks, defined
%% as GreenNotebook macros, get assigned in \MYgreens.
%%
%% This file doubles as a latex'able example in-game notebook.
%% README-green.tex, also in this directory, doubles as an example
%% out-of-game notebook.
%%
%% _template.tex serves as a bare-bones template suitable for
%% copying when starting a new notebook.
%%
%% Notebook and GreenNotebook macros each have a file that lives here.
%% The argument to \startnotebook{...}  probably should be the macro
%% for the given notebook.  However, you can also just use
%% \startnotebook{Some Text} if you want.
%%
%% Note that every \startnotebook command needs a matching
%% \endnotebook command.  Also note that no ownership information
%% appears on the notebook.
%%
%%
%% If you want to use notebooks for memory packets, see related macros
%% in Lists/mem-LIST.tex (and in Charsheets/README.tex).  If you want
%% more programmatic control over the structure/linkage of notebooks,
%% like for a computer security mechanic, see LabelCover and LabelPage
%% in Lists/sign-LIST.tex.
%%
%%%%%

\documentclass[notebook]{guildcamp1} %% [notebook] or [greennotebook]
\begin{document}

\startnotebook{\nTest{}}



%% The argument to \begin{page}{...} is a reference string.  You use
%% \nbref{...} in page text to refer to a given page.  These
%% references will be typeset as numbers, based on the order of the
%% pages.
\begin{page}{first}

Get together 3 people who know how to juggle.  Then get 9 juggling
balls.  Have each juggler juggle three at once, such that all 9 are
being juggled at the same time.

If the group can go 1 minute or more without dropping a single ball,
go to step \nbref{1min}.  If they can go for 5 minutes without dropping
a single ball, go to step \nbref{good}.

\end{page}



%% the optional argument [One Minute] means this page will be labeled and
%% referenced with a string instead of a number.  The numbering of other
%% pages will ignore this one.
\begin{page}[One Minute]{1min}

blah you need to juggle better.

\end{page}



\begin{page}{nothing}

blah

\end{page}



\begin{page}{good}

blah not bad.  Continue for another 5 minutes, and you can go to step
\nbref{last}.

\end{page}



\begin{page}{last}

blah you win!  congratulations!

\end{page}



\endnotebook

\end{document}
